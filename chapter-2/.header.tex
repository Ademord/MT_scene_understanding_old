\chapter{Foundations}\label{chap2:title}
% \section{Background and Related Work}
% 3 pages max, i was expecting 
This chapter introduces the related efforts and theory required to understand the proposed solution to the research questions. 
Section \ref{chap2:related-work} describes the related works to this thesis in fields such as robotic navigation and exploration, 3D vision, active vision, etc.
% Afterwards, Section \ref{chap2:compvis} presents briefly the genesis of computer vision and the rationale for developing computer vision techniques. These concepts are then followed by the benefit 
Afterwards, Section \ref{chap2:voxels} introduces the concept of voxels as 3D data structures that reduce complexity in a scene, 
while Section \ref{chap2:octrees} covers the efficiency provided through octrees in the manipulation of 3D volumes. 
The limitations of current 3D scene understanding techniques are presented in Section \ref{chap2:scene-understanding} and used for motivation in Section \ref{chap2:3denvironment}, which goes over the promise and exploitation of 3D environments and synthetic data using game engines and simulation environments.
% covers the base concepts of game engines, simulation environments and sheds light on the promise of synthetic data. 
% Section \ref{chap:2:machinelearning} introduces the breakthroughs in computer vision thanks to deep learning techniques. 
% Lastly, it describes 
% introduces the breakthroughs in computer vision thanks to deep learning techniques.
% \ref{chap2:3denvironment}
% Narrowing the scope, section
% \ref{chap:2:3d} the advancements in 3D object recognition and its limitations, 
% and section \ref{chap:2:semantic-segmentation} sheds a light on state of the art for object segmentation techniques. 
% Section \ref{chap:2:pose} presents the topic of pose estimation, the current techniques and its limitations. 
Finally, section \ref{chap2:reinforcement-learning} introduces the basics of reinforcement learning, Q-learning, how function approximators allow agents to learn more sophisticated environments and wraps up the theoretical background with proximal policy optimization.
% summarizes the computer vision timeline and gives an overall idea of the path research has taken up to today for 3D pose estimation. 

% The following questions will be answered in this chapter:
% \begin{itemize}
%     % \item What are the origins of computer vision?
%     % \item What propelled the development of deep learning techniques in computer vision?
%     % \item What is the current state of 3D object model estimation techniques?
%     % \item What is the current state of pose estimation techniques?
%     \item What are the related efforts?
%     \item What are voxels and how are they constructed?
%     \item What are octrees?
%     \item How are 3D environments fabricated?
%     \item What can simulation engines do?
%     \item What is synthetic data?
%     % is the current state of
%     \item What is reinforcement learning, function approximation and PPO?
% \end{itemize}
